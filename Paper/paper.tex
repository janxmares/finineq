%!TEX program = xelatex
%!BIB program = bibtex

\documentclass[a4paper,11pt]{article}
\usepackage{graphicx}
\usepackage{amsmath, amsthm, amssymb} %better typesetting of mathematical expressions, theorems, additional symbols
\usepackage{booktabs} %fixltx2e
\usepackage[flushleft]{threeparttable}
\usepackage{tabularx}
% \usepackage[capposition=top]{floatrow}
\usepackage{caption}
\captionsetup[measuredfigure]{labelformat=empty}
% \usepackage{bm}
% \usepackage{subcaption}
%\usepackage{epstopdf} % not supported by xelatex
% \usepackage{fullpage}
% \usepackage{enumerate}
\usepackage{authblk}
\usepackage{fontspec}
\usepackage{lscape}
\usepackage[round]{natbib} %standard package for bibliography
\usepackage{multirow}
% \usepackage[figuresleft]{rotating}
\usepackage[dvipsnames]{xcolor}
\usepackage{colortbl} % color table
\usepackage{geometry}
% \usepackage{hyperref} %hyperlink support
% \hypersetup{pdfstartview={XYZ null null 1.00},bookmarksnumbered, hypertexnames=false, colorlinks=true, linkcolor=BrickRed, citecolor=MidnightBlue, urlcolor=MidnightBlue} % zoom: default screen; numbers for bookmarks

%
%
%\usepackage{parskip}
\usepackage[nolist]{acronym}
\usepackage{longtable}
\usepackage{setspace}

\acrodef{bma}[BMA]{Bayesian Model Averaging}
\acrodef{bic}[BIC]{Bayesian Information Criterion}
\acrodef{cbf}[CBF]{Conditional Bayes Factor}
\acrodefplural{cbf}[CBFs]{Conditional Bayes Factors}
\acrodef{cswd}[CSWD]{Credit Suisse Wealth Databook}
\acrodef{gfdd}[GFDD]{Global Financial Development Database}
\acrodef{pip}[PIP]{Posterior Inclusion Probability}
\acrodefplural{pip}[PIPs]{Posterior Inclusion Probabilities}
\acrodef{fdi}[FDI]{Foreign Direct Investment}
\acrodefplural{fdi}[FDIs]{Foreign Direct Investments}
\acrodef{gdp}[GDP]{Gross Domestic Product}
\acrodef{uip}[UIP]{Unit Information Prior}
\acrodef{wb}[WB]{World Bank}
\acrodef{oecd}[OECD]{Organisation for Co-operation and Development}
\acrodef{ivbma}[IVBMA]{Instrumental Variable Bayesian Model Averaging}
\acrodef{wid}[WID]{World Inequality Database}
\acrodef{hbs}[HBS]{Household Balance Sheet}
\acrodef{uk}[UK]{United Kingdom}
\acrodef{us}[US]{United States}
\acrodef{efw}[EFW]{Economic Freedom of the World}
\acrodef{pmp}[PMP]{Posterior Model Probability}

%
%

\setlength\parindent{34pt}
\onehalfspacing
\title{Finance and Income Inequality - panel BMA approach\thanks{We thank two anonymous referees, ...}}
\author[a]{Roman Horvath}
\author[a]{Jan Mares\footnote{\footnotesize Corresponding author's address: IES FSV UK, Opletalova 26, 110 00, Praha 1; \textbf{e-mail:janxmares@gmail.com}}}
\affil[a]{Charles University, Prague}


%\renewcommand\Authands{ and }
\date{\today}

\begin{document}


\clubpenalty 9999 % no orphants (typographic properties)
\widowpenalty 9999 % no widows (typographic properties)
\def\sym#1{\ifmmode^{#1}\else\(^{#1}\)\fi} % shortcut for Stata tables

\maketitle

\thispagestyle{empty}
\begin{abstract}
Using a global sample, this paper investigates the determinants of wealth inequality capturing various economic, financial, political, institutional, and geographical indicators. Using instrumental variable Bayesian model averaging, it reveals that only a handful of indicators robustly matters and finance plays a key role. It reports that while financial depth increases wealth inequality, efficiency and access to finance reduce inequality. In addition, redistribution and education are associated with lower inequality whereas wars and openness to international trade contribute to greater wealth inequality.
\end{abstract}

\bigskip

\begin{tabular}{p{0.25\hsize}p{0.6\hsize}} %0.15
\textbf{Keywords:} & Income inequality, finance, Bayesian model averaging
\end{tabular}

\bigskip

\begin{tabular}{p{0.25\hsize}p{0.6\hsize}}
\textbf{JEL Codes:} & D31, E21\\
\end{tabular}

\clearpage
\setcounter{page}{1}

\section{Introduction}
% \label{sec:intro}

\section{Related literature}
% \label{sec:literature}
\citet{pikettyandzucman2014}
\citet{van2018inequality} provide evidence that income inequality in the \ac{us} has different implications for the future income growth of the rich and the poor. High inequality seems to hurt the prospects of the poor while the top for the distribution is unaffected. The rich thus disproportianately benefit from higher inequality as their subsequent income exhibit faster growth. The authors attribute this effect to the political channel the rich use to lobby for implementation for the policies which back their economic interests. Preferences of the rich  are ultimately more likely to determine public policy than the preferences of the majority \citep{gilens_page_2014}. High inequality together with a credit constraint and rich driving the political process results in low government spending and lasting high inequality.

\citet{marrero2013inequality} introduce two types of inequality - inequality of oppportunity and inequality of effort/luck. Applying this to the EU countries and US states, they show that inequality of opportunity (driven by race, parental education, citizenship, ...) is negatively related to growth while the residual "good" inequality is growth inducing.

\section{Data}
% \label{sec:data}

\section{Methodology}
% \label{sec:meth}

\section{Results}
% \label{sec:results}

% \label{sec:conclusion}

\clearpage
%
\bibliographystyle{chicagoa}
\bibliography{literature}
%
\clearpage
%
\appendix
\section{Appendix}
% \label{sec:app}

\renewcommand{\thesection}{A\arabic{section}}%
\renewcommand{\thetable}{A\arabic{table}}%
\renewcommand{\thefigure}{A\arabic{figure}}%
\renewcommand{\theequation}{A\arabic{eq}} 

\end{document}